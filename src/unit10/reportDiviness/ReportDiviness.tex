\documentclass{article}
\usepackage[top=1in, bottom=1in, left=1in, right=1in]{geometry}

\begin{document}
\noindent
Austin Diviness \\
3/25/2014 \\
Web Apps \\
Haml vs erb \\

\paragraph{ERB} 
ERB stands for Embedded RuBy, it is a templating system modeled after HTML and PHP embedding. Ruby style code is held in special tags inside an HTML document. \texttt{<\% \%>} tags evaluate the expression they contain, while \texttt{<\%= \%>} tags evaluate their expression and print the result to the resulting output file. 


\paragraph{HAML} 
HAML stands for HTML Abstract Markup Language, it is a templating system similar to Sass for CSS files. Instead of embedding special tags in a document, Haml instead is a description of the entire document using its markup language. A \texttt{\%} followed by the tag name describes a simple tag in Haml, with an optional ruby style hash directly following it to describe any tag attributes. The hash is of the form \texttt{symbol => string}. A \texttt{=} can be used at the end of a Haml tag description to indicate that the tag has ruby code that must be evaluated. CSS style \texttt{\#} and \texttt{.} can be used to indicate the id and class of a tag, instead of using the tag description hash. Tags can be nested with further indentation levels, like in Python.

\paragraph{Comparision} 
ERB acts more as a supplement to an existing markup file then Haml, allowing web designers that are already familiar with HTML and CSS to pick up ERB quicker. Haml allows for compactness in a markup file, eliminating the need for closing tags and other HTML verboseness, at the expense of learning a new markup language. Because Haml is a new language rather than a couple new tags, those unfamiliar with it will have a more difficult time reading and understanding Haml files.

\paragraph{Examples} 
ERB \\
\begin{verbatim}
<div id="content">
    <div class="left_colume">
        <h2>Welcome to our site!</h2>
        <p><%= print_information %></p>
    </div>
    <div class="right_colume">
        <%= render :partial => "sidebar" %>
    </div>
</div>
\end{verbatim}

Haml
\begin{verbatim}
#content
    .left_column
        %h2 Welcome to our site!
        %p= print_information
    .right_column
        = render :partial => 'sidebar'
\end{verbatim}

\end{document}
